%% March 2018
%%%%%%%%%%%%%%%%%%%%%%%%%%%%%%%%%%%%%%%%%%%%%%%%%%%%%%%%%%%%%%%%%%%%%%%%%%%%
% AGUJournalTemplate.tex: this template file is for articles formatted with LaTeX
%
% This file includes commands and instructions
% given in the order necessary to produce a final output that will
% satisfy AGU requirements, including customized APA reference formatting.
%
% You may copy this file and give it your
% article name, and enter your text.
%
%
% Step 1: Set the \documentclass
%
% There are two options for article format:
%
% PLEASE USE THE DRAFT OPTION TO SUBMIT YOUR PAPERS.
% The draft option produces double spaced output.
%

%% To submit your paper:
\documentclass[draft,linenumbers]{agujournal2018}
\usepackage{apacite}
\usepackage{url} %this package should fix any errors with URLs in refs.
%%%%%%%
% As of 2018 we recommend use of the TrackChanges package to mark revisions.
% The trackchanges package adds five new LaTeX commands:
%
%  \note[editor]{The note}
%  \annote[editor]{Text to annotate}{The note}
%  \add[editor]{Text to add}
%  \remove[editor]{Text to remove}
%  \change[editor]{Text to remove}{Text to add}
%
% complete documentation is here: http://trackchanges.sourceforge.net/
%%%%%%%


%% Enter journal name below.
%% Choose from this list of Journals:
%
% JGR: Atmospheres
% JGR: Biogeosciences
% JGR: Earth Surface
% JGR: Oceans
% JGR: Planets
% JGR: Solid Earth
% JGR: Space Physics
% Global Biogeochemical Cycles
% Geophysical Research Letters
% Paleoceanography and Paleoclimatology
% Radio Science
% Reviews of Geophysics
% Tectonics
% Space Weather
% Water Resources Research
% Geochemistry, Geophysics, Geosystems
% Journal of Advances in Modeling Earth Systems (JAMES)
% Earth's Future
% Earth and Space Science
% Geohealth
%
% ie, \journalname{Water Resources Research}

\journalname{JGR: Atmospheres}

\usepackage{gensymb}
\usepackage{soulutf8}

\begin{document}

%% ------------------------------------------------------------------------ %%
%  Title
%
% (A title should be specific, informative, and brief. Use
% abbreviations only if they are defined in the abstract. Titles that
% start with general keywords then specific terms are optimized in
% searches)
%
%% ------------------------------------------------------------------------ %%

% Example: \title{This is a test title}

\title{Are planetary waves the same thing as quasi-stationary waves?}

%% ------------------------------------------------------------------------ %%
%
%  AUTHORS AND AFFILIATIONS
%
%% ------------------------------------------------------------------------ %%

% Authors are individuals who have significantly contributed to the
% research and preparation of the article. Group authors are allowed, if
% each author in the group is separately identified in an appendix.)

% List authors by first name or initial followed by last name and
% separated by commas. Use \affil{} to number affiliations, and
% \thanks{} for author notes.
% Additional author notes should be indicated with \thanks{} (for
% example, for current addresses).

% Example: \authors{A. B. Author\affil{1}\thanks{Current address, Antartica}, B. C. Author\affil{2,3}, and D. E.
% Author\affil{3,4}\thanks{Also funded by Monsanto.}}

\authors{
Elio Campitelli
\affil{1}
Carolina Vera
\affil{1, 2}
Leandro Díaz
\affil{1, 2}
}


% \affiliation{1}{First Affiliation}
% \affiliation{2}{Second Affiliation}
% \affiliation{3}{Third Affiliation}
% \affiliation{4}{Fourth Affiliation}

\affiliation{1}{Centro de Investigaciones del Mar y la Atmosfera, UMI-IFAECI
(CONICET-UBA-CNRS)}
\affiliation{2}{Departamento de Ciencias de la Atmósfera y los Océanos (FCEyN, UBA)}
%(repeat as many times as is necessary)

%% Corresponding Author:
% Corresponding author mailing address and e-mail address:

% (include name and email addresses of the corresponding author.  More
% than one corresponding author is allowed in this LaTeX file and for
% publication; but only one corresponding author is allowed in our
% editorial system.)

% Example: \correspondingauthor{First and Last Name}{email@address.edu}
\correspondingauthor{Elio Campitelli}{elio.campitelli@cima.fcen.uba.ar}

%% Keypoints, final entry on title page.

%  List up to three key points (at least one is required)
%  Key Points summarize the main points and conclusions of the article
%  Each must be 100 characters or less with no special characters or punctuation

% Example:
% \begin{keypoints}
% \item	List up to three key points (at least one is required)
% \item	Key Points summarize the main points and conclusions of the article
% \item	Each must be 100 characters or less with no special characters or punctuation
% \end{keypoints}

\begin{keypoints}
\item Zonal waves and Quasi-stationary waves are disctinct but related
phenomena
\item This distinction has theoretical and practical implications
\item The relationship between the mean ZW amplitude and QS amplitude yields
an estiamte of stationarity
\end{keypoints}

%% ------------------------------------------------------------------------ %%
%
%  ABSTRACT
%
% A good abstract will begin with a short description of the problem
% being addressed, briefly describe the new data or analyses, then
% briefly states the main conclusion(s) and how they are supported and
% uncertainties.
%% ------------------------------------------------------------------------ %%

%% \begin{abstract} starts the second page

\begin{abstract}
In the meteorological literature the analysis of the zonally asymmetric
it is very common to analyse
\end{abstract}

\section{Introduction}

Many atmospheric variables have a strong dependence with latitude, so it
is often natural to decompose them into a zonal mean component and a
deviation from it. If \(\phi\) is a generic variable, then

\begin{linenomath*}
\begin{equation}\label{eq:Z}
\phi_{(x, y, z, t)} = [\phi]_{(y, z, t)} + \phi_{(x, y, z, t)}^*
\end{equation}
\end{linenomath*}

where \([\phi]\) is the mean zonal field and \(\phi^*\), the deviations
from it. This zonally asymmetric part is sometimes called ``zonal wave''
or ``planteary wave''. The names ``stationary wave'' or
``quasi-strationary wave'', on the other hand, are generaly reserved to
the zonal asymmetires of the time mean field (\(\overline{\phi}^*\)).
However, these terms are sometimes used interchangeably in the
literature (e.g. \citet{Rao2004}, \citet{Raphael2004},
\citet{Kravchenko2012}, \citet{Irving2015}, \citet{Turner2017},
\citet{Lastovicka2018}) which could lead to some confusion.

Given a set of atmospheric fields, we define \emph{zonal waves} (ZW) as
waves observed in each individual ``instantaneous'' field and
\emph{quasi-stationary waves} (QS) as the resulting waves in the mean
field. While these definitions depend on which are the ``instantaneous
field'' in question (monthly, daily, subdaily, etc\ldots{}) and the
averaging timescales used, they illustrate that ZWs are properties of
the \emph{elements} of the set, while the QSs are properties of the set
as a whole. This is an important distinction with theoretical and
methodological implications that is not always appreciated in the
literature.

\section{Story}

\begin{figure}[h]

{\centering \includegraphics{fig/QS-ZW/rao-1} 

}

\caption{Seasonal cycle of amplitude of the geopotential planetary waves 1 to 3 at 60\degree S computed as the mean amplitude of the monthly waves ($\overline{ZW}$) and as the amplitude of the mean wave (QS). The period of analysis is 1950 to 1998. The left column reproduces Figure 3 from @Rao2004.}\label{fig:rao}
\end{figure}

To illustrate the distinction between ZWs and QS, Figure \ref{fig:rao}
shows the monthly seasonal cycle of amplitude of planetary waves at
60\degree S using monthly fields from the NCEP/NCAR reanalysis
(\citet{Kalnay1996}) between 1950 and 1998. The left column
(\(\overline{ZW}\)) reproduces Figure 3 from \citet{Rao2004} and is
computed by taking --for each month and level-- the average amplitude of
the 49 individual amplitudes. The right column (QS), on the other hand,
is computed by taking the amplitude of the average geopotential field
for each month and level.

The resulting fields convey different information. First, the amplitude
of \(\overline{ZW}\) fields is always greater than the one for QS
fields. This is a mathematical necessity (\emph{xx¿Deberia demostrar
eso? Vale la pena una demostracion en un material suplementario?xx})
that explains \citet{Rao2004}'s observation that their Wave 1 amplitude
was greater than that reported by \citet{Hurrell1998}. Secondly, they
have different annual cycles and vertical structures. QS2, for example,
has a strong minimum in the low stratosphere during the austral autumn
that is not apparent in \(\overline{ZW2}\). Similarly, the austral
winter mid-tropospheric maximum is very well defined in
\(\overline{ZW3}\) but not so in QS3. Thirdly, the relative importance
between each wave number vary. \(\overline{ZW}\) fields show, for
example, a preponderance of wave 2 over 3 in almost every level and
month. However, the QS3 has greater amplitude than QS2 in the first half
of the year. In contrast with wave-numbers 2 and 3, \(\overline{ZW1}\)
and QS1 fields are very similar.

\begin{figure}[h]

{\centering \includegraphics{fig/QS-ZW/hurrell-1} 

}

\caption{Seasonal cycle of amplitude of the geopotential planetary waves 2 at 300hPa computed as the mean amplitude of the monthly waves ($\overline{ZW}$) and as the amplitude of the mean wave (QS). From monthly NCEP/NCAR Reanalysis, 1979 to 2017.}\label{fig:hurrell}
\end{figure}

These differences are related to the degree of stationarity of zonal
waves and are location-dependent. Figure \ref{fig:hurrell} show the same
variable that Figure \ref{fig:rao} but for 300hPa. The contrast between
the northern and southern hemisphere is not only evident in the
amplitude of the planetary waves, but also in the comparison between
\(\overline{ZW}\) and QS. Specially for wave-numbers 2 and 3,
\(\overline{ZW}\) and QS fields are very similar in the north but they
have significant differences in the south.

\subsection{Stationarity}

Another important consequence of the distinction between
\(\overline{ZW}\) and QS is that the quotient between the two can be
used as a measure of stationarity. As an analogy with the constancy of
the wind (\citet{Singer1967}), planetary wave stationarity can be
estimated as

\begin{linenomath*}
\begin{equation}\label{eq:S}
\hat{S} =  \frac{QS}{\overline{ZW}}
\end{equation}
\end{linenomath*}

It can be shown that \(\hat{S} = 1\) for completely stationary waves and
that \(E(\hat{S}) = n^{-1/2}\) for completely non-stationary waves
(where \(n\) is the sample size).

\begin{figure}[h]

{\centering \includegraphics{fig/QS-ZW/stationarity-1} 

}

\caption{Seasonal cycle of stationarity of the 300hPa geopotential QS2 computed using Equation \ref{eq:S} (shaded) and $\overline{ZW2}$ (contours). From monthly NCEP/NCAR Reanalysis, 1979 to 2017.}\label{fig:stationarity}
\end{figure}

As an example, Figure \ref{fig:stationarity} shows \(\hat{S}\) for QS2
computed using Equation \ref{eq:S}. The southern hemisphere clearly
shows a lower degree of QS2 stationarity than the northern hemisphere or
the tropics. In the northern mid latitudes there is a seasonal cycle of
stationarity that follows the seasonal cycle of \(\overline{ZW}\)
(Figure \ref{fig:hurrell}). In the southern hemisphere, instead, the
June maximum of \(\overline{ZW}\) at 60\degree S coincides with a
minimum of stationarity.

While \(\hat{S}\) is used --sometimes with a \(\arcsin\) transformation
(\citet{Singer1967})-- in the meteorological literature in the context
of wind steadiness, to our knowledge this is the first time it has been
applied to the study of atmospheric waves. Furthermore, its statistical
properties are not well studied. For example, it can be seen that the
estimation of \(\hat{S}\) from a finite sample has a positive bias that
is inversely proportional to the population stationarity, but its
convergence properties are not explored.

\subsection{QS activity}

Defining quasi-stationary waves as a climatological property of a set of
atmospheric fields, precludes, in principle, the possibility of
quantifying a QS metric that applies to instantaneous fields. It would
seem impossible to, for example, construct an time series of QS activity
that could be use as a basis for correlations with other variables,
compositions or for use in other methodologies. But there are ways of
solving this issue.

One possibility is recognising that individual fields can be
characterised by their degree of similarity with the climatological QS.
The index produced by \citet{Raphael2004} for the QS3 is an example.
While not expressly a measure of similarity, it is sensitive to wave 3
patterns with phase close to the stationary phase. \citet{Yuan2008} use
Principal Component Analysis on the meridional wind field; the spatial
pattern of the leading mode is very similar to the QS3 and so a time
series can be obtained by projecting each instantaneous field to it.

Another way of constructing a time series is to exploit the fact the
timescale dependence of QS. By applying a running mean with a suitable
window before computing wave amplitudes, one obtains the QS wave
amplitude of that window. This is the methodology applied by
\citet{Wolf2018} who performed a 15 day low pass filter before computing
wave envelopes. Each data time represents, then, the mean field of a set
of fields inside the 15 day window an thus waves computed from them are
actually QS waves for each of those sets. (\emph{xx no estoy seguro que
se entienda bien xx})

\section{Conclusions}

The fact that zonal waves (ZW) and quasi-stationary waves (QS) are two
distinct but related phenomena has both practical and theoretical
implications. First, it underscores the importance of

Researchers should be aware of which phenomena they want to study and
use the appropriate methods. The mean amplitude of the ZW could be
appropireate to study the vertical propagation of Rossby waves, for
example. But ZW amplitude could lead to misleading results if used as
the basis of local impacts studies because they are probably more
influenced by phase effects. For clarity and reproducibility, we
encourage researchers in the field to describe if they are using the
mean amplitude of the individual waves or the amplitude of the mean
wave.

Comparison between results should also be made having this issues in
mind. For example, \citet{Irving2015} compare their planetary wave
activity index with \citet{Raphael2004}'s wave 3 index and conclude that
the later cannot account for events with waves far removed from their
climatological position. However, if we understand it as an index of QS3
similitude, then it is a feature, not a bug.

Since planetary waves are generally more stationary in the northern
hemisphere, these issues are more critical in studies of the southern
hemisphere.

Besides those direct implications, separating ZW and QS can lead to
novel levels of analysis. Here, we showed a simple metric of the
stationarity, but others applications are also possible. For example,
\citet{Smith2012} showed that linear interference between the QS1 and
ZW1 was related to vertical wave activity transport at the troposphere.

\bibliography{qszw.bib}


\end{document}
