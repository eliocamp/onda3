%% March 2018
%%%%%%%%%%%%%%%%%%%%%%%%%%%%%%%%%%%%%%%%%%%%%%%%%%%%%%%%%%%%%%%%%%%%%%%%%%%%
% AGUJournalTemplate.tex: this template file is for articles formatted with LaTeX
%
% This file includes commands and instructions
% given in the order necessary to produce a final output that will
% satisfy AGU requirements, including customized APA reference formatting.
%
% You may copy this file and give it your
% article name, and enter your text.
%
%
% Step 1: Set the \documentclass
%
% There are two options for article format:
%
% PLEASE USE THE DRAFT OPTION TO SUBMIT YOUR PAPERS.
% The draft option produces double spaced output.
%

%% To submit your paper:
\documentclass[draft,linenumbers]{agujournal2018}
\usepackage{apacite}
\usepackage{url} %this package should fix any errors with URLs in refs.
%%%%%%%
% As of 2018 we recommend use of the TrackChanges package to mark revisions.
% The trackchanges package adds five new LaTeX commands:
%
%  \note[editor]{The note}
%  \annote[editor]{Text to annotate}{The note}
%  \add[editor]{Text to add}
%  \remove[editor]{Text to remove}
%  \change[editor]{Text to remove}{Text to add}
%
% complete documentation is here: http://trackchanges.sourceforge.net/
%%%%%%%


%% Enter journal name below.
%% Choose from this list of Journals:
%
% JGR: Atmospheres
% JGR: Biogeosciences
% JGR: Earth Surface
% JGR: Oceans
% JGR: Planets
% JGR: Solid Earth
% JGR: Space Physics
% Global Biogeochemical Cycles
% Geophysical Research Letters
% Paleoceanography and Paleoclimatology
% Radio Science
% Reviews of Geophysics
% Tectonics
% Space Weather
% Water Resources Research
% Geochemistry, Geophysics, Geosystems
% Journal of Advances in Modeling Earth Systems (JAMES)
% Earth's Future
% Earth and Space Science
% Geohealth
%
% ie, \journalname{Water Resources Research}

\journalname{Geophysical Research Letters}

\usepackage{gensymb}
\usepackage{soulutf8}

\begin{document}

%% ------------------------------------------------------------------------ %%
%  Title
%
% (A title should be specific, informative, and brief. Use
% abbreviations only if they are defined in the abstract. Titles that
% start with general keywords then specific terms are optimized in
% searches)
%
%% ------------------------------------------------------------------------ %%

% Example: \title{This is a test title}

\title{How stationary are planetary waves in the Southern Hemisphere?}

%% ------------------------------------------------------------------------ %%
%
%  AUTHORS AND AFFILIATIONS
%
%% ------------------------------------------------------------------------ %%

% Authors are individuals who have significantly contributed to the
% research and preparation of the article. Group authors are allowed, if
% each author in the group is separately identified in an appendix.)

% List authors by first name or initial followed by last name and
% separated by commas. Use \affil{} to number affiliations, and
% \thanks{} for author notes.
% Additional author notes should be indicated with \thanks{} (for
% example, for current addresses).

% Example: \authors{A. B. Author\affil{1}\thanks{Current address, Antartica}, B. C. Author\affil{2,3}, and D. E.
% Author\affil{3,4}\thanks{Also funded by Monsanto.}}

\authors{
Elio Campitelli
\affil{1}
Carolina Vera
\affil{1, 2}
Leandro Díaz
\affil{1, 2}
}


% \affiliation{1}{First Affiliation}
% \affiliation{2}{Second Affiliation}
% \affiliation{3}{Third Affiliation}
% \affiliation{4}{Fourth Affiliation}

\affiliation{1}{Centro de Investigaciones del Mar y la Atmosfera, UMI-IFAECI
(CONICET-UBA-CNRS)}
\affiliation{2}{Departamento de Ciencias de la Atmósfera y los Océanos (FCEyN, UBA)}
%(repeat as many times as is necessary)

%% Corresponding Author:
% Corresponding author mailing address and e-mail address:

% (include name and email addresses of the corresponding author.  More
% than one corresponding author is allowed in this LaTeX file and for
% publication; but only one corresponding author is allowed in our
% editorial system.)

% Example: \correspondingauthor{First and Last Name}{email@address.edu}
\correspondingauthor{Elio Campitelli}{elio.campitelli@cima.fcen.uba.ar}

%% Keypoints, final entry on title page.

%  List up to three key points (at least one is required)
%  Key Points summarize the main points and conclusions of the article
%  Each must be 100 characters or less with no special characters or punctuation

% Example:
% \begin{keypoints}
% \item	List up to three key points (at least one is required)
% \item	Key Points summarize the main points and conclusions of the article
% \item	Each must be 100 characters or less with no special characters or punctuation
% \end{keypoints}

\begin{keypoints}
\item Zonal waves and Quasi-stationary waves are disctinct but related
phenomena
\item This distinction has theoretical and practical implications
\item The relationship between the mean ZW amplitude and QS amplitude yields
an estimate of stationarity
\end{keypoints}

%% ------------------------------------------------------------------------ %%
%
%  ABSTRACT
%
% A good abstract will begin with a short description of the problem
% being addressed, briefly describe the new data or analyses, then
% briefly states the main conclusion(s) and how they are supported and
% uncertainties.
%% ------------------------------------------------------------------------ %%

%% \begin{abstract} starts the second page

\begin{abstract}
In the meteorological literature the analysis of the zonally asymmetric
it is very common to analyse
\end{abstract}

\section{Introduction}

Zonal waves, also called planetary waves, that can develop in the
extratropical latitudes of the Southern Hemisphere (SH), have received
some attention by the scientific community because of its role in
modulating weather systems and regional climate (xxREF). They are
typically characterized by applying Fourier decomposition to hemispheric
anomalies of sea-level pressure of geopotential heights. On the other
hand, ``stationary waves'' or ``quasi-stationary waves'' are terms
generally reserved in the literature to the zonal asymmetries of the
time mean field (\(\overline{\phi}^*\)). These terms are sometimes used
interchangeably in the SH circulation related studies
\citep[e.g.][]{Rao2004, Raphael2004, Kravchenko2012, Irving2015, Turner2017, Lastovicka2018}.

\emph{xx me parece que la explicación de cada uno tiene que venir
después de las definiciones de la siguiente sección}

However, it is not evident from the current knowledge, how
``stationary'' or ``quasi-stationary'' the zonal waves are in the SH.
The focus of this study is then to assess the xx\ldots{}me preocupa que
haya papers olvidados sobre este tema.

\section{Zonal waves and quasi-stationary waves}

In this study we define \emph{planetary waves} as waves that encompass a
full latitude circle. Planetary waves of the ``instantaneous'' fields
will be called \emph{zonal waves} (ZW) and the ones of the field mean
will be called \emph{quasi-stationary waves} (QS). They are
characterised by their wavenumber, amplitude and phase such that

\begin{linenomath*}
\begin{eqnarray}\label{eq:ZW}
\mathrm{ZWk}(t) & = & A_\mathrm{ZWk}(t)\cos \left [ k\lambda - \alpha_\mathrm{ZWk}(t) \right ] \\ 
\overline{\mathrm{ZWk}(t)} = \mathrm{QSk} & = & A_\mathrm{QSk}\cos \left (  \mathrm{k}\lambda - \alpha_\mathrm{QSk} \right ) \label{eq:QS}
\end{eqnarray}
\end{linenomath*}

where \(\mathrm{k}\) is the wavenumber, \(\lambda\) the longitude, and
\(\mathrm{A_{x}}\) and \(\alpha_\mathrm{x}\), the amplitude and phase of
each wave. Note that \(\mathrm{ZWk}(t)\) is made expressly dependent on
time, while \(\mathrm{QSk}\) is not. Furthermore, from the properties of
wave superposition it can be seen that, in general,
\(\alpha_\mathrm{QSk}\) does not equal
\(\overline{\alpha_\mathrm{ZWk}}\) an that \(A_\mathrm{QSk}\) will
always be less or equal than \(\overline{A_\mathrm{ZWk}}\)
\citep{Pain2005}.

While these definitions depend on which are the ``instantaneous field''
in question (monthly, daily, sub daily, etc\ldots{}) and the averaging
time scale, they illustrate that ZW are properties of the
\emph{elements} of the set, while QS are properties of the set as a
whole. This is an important distinction with theoretical and
methodological implications that is not always differentiated in the
literature.

For example, while \citet{Quintanar1995a} use the term
``quasi-stationary waves (QS)'' to refer to geopotential QS as defined
by Equation \ref{eq:QS}, \citet{Raphael2004} developed an index of QS3
but uses the term ``zonal wave (ZW)'' in her description. This change in
naming convention is not recognized by \citet{Irving2015}, who compare
\citet{Raphael2004}'s QS index with their own index of southern
hemisphere ZW amplitude.

\citet{Rao2004}, on the other hand, follow the nomenclature from
\citet{Quintanar1995a} for QS, but in their exploration of its
climatology, they use \(\overline{A_\mathrm{ZWk}}\) instead of
\(A_\mathrm{QSk}\). \citet{Kravchenko2012} do the same in the context of
air temperature. \citet{Turner2017} use the terms ``planetary wave
\emph{k}'', ``quasi-stationary wave \emph{k}'' and ``wave number
\emph{k}'' to refer to \(\mathrm{QSk}\), but they analyse
\(A_\mathrm{ZWk}\) and \(\alpha_\mathrm{ZWk}\). Finally,
\citet{Lastovicka2018} study QS and ZW but they use the term
``stationary planetary wave (SPW)'' to refer to both.

\begin{figure}[h]

{\centering \includegraphics{fig/QS-ZW/rao-1} 

}

\caption{Seasonal cycle of amplitude of the geopotential planetary waves 1 to 3 at 60\degree S computed as the amplitude of the mean wave ($A_\mathrm{QSk}$) and as the mean amplitude of the monthly waves ($\overline{A_\mathrm{ZW} }$).}\label{fig:rao}
\end{figure}

Figure \ref{fig:rao} shows the seasonal cycle of the amplitude of
planetary waves at 60\degree S using monthly fields from the NCEP/NCAR
reanalysis \citep{Kalnay1996} between 1950 and 1998. The left column
(\(A_\mathrm{QS}\)) is computed by taking the amplitude of the averaged
geopotential field for each month, level and wavenumber. The right
column (\(\overline{A_\mathrm{ZW}}\)) is computed by taking the average
amplitude of the 49 individual ZW.

Figure \ref{fig:rao} shows that both amplitudes have different annual
cycles and vertical structures. \(A_\mathrm{QS2}\) has a strong minimum
in the low stratosphere during the austral autumn that is not apparent
in \(\overline{A_\mathrm{ZW2}}\). Similarly, the austral winter
mid-tropospheric maximum is very well defined in
\(\overline{A_\mathrm{ZW3}}\) but not so in \(A_\mathrm{QS3}\). The
relative individual contribution of each wavenumber is also different.
\(\overline{A_\mathrm{ZW}}\) fields shows a preponderance of wave 2 over
3 in almost every level and month. However, \(A_\mathrm{QS3}\) is larger
than \(A_\mathrm{QS2}\) in the first half of the year. In contrast with
wavenumbers 2 and 3, \(\overline{A_\mathrm{ZW1}}\) and
\(A_\mathrm{QS1}\) fields are very similar.

\begin{figure}[h]

{\centering \includegraphics{fig/QS-ZW/hurrell-1} 

}

\caption{Seasonal cycle of amplitude of the geopotential planetary waves 2 at 300hPa computed as the amplitude of the mean wave ($A_\mathrm{QSk}$) and as the mean amplitude of the monthly waves ($\overline{A_\mathrm{ZW} }$).}\label{fig:hurrell}
\end{figure}

These differences are location-dependent. Figure \ref{fig:hurrell} show
the horizontal distribution of \(A_\mathrm{QS}\) and
\(\overline{A_\mathrm{ZW}}\) at 300hPa, for the three wavenumbers
considered. In the northern hemisphere there is a strong seasonal cycle
of \(A_\mathrm{QS}\) that is matched by the seasonal cycle of
\(\overline{A_\mathrm{ZW}}\) for all wavenumbers. In contrast, in the
southern hemisphere the seasonal cycles of \(A_\mathrm{QS}\) and
\(\overline{A_\mathrm{ZW}}\) are similar only for wavenumber 1.
Wavenumbers 2 and 3 have much lower \(A_\mathrm{QS}\) than
\(\overline{A_\mathrm{ZW}}\).

\subsection{Stationarity Index}

\citet{Loon1972} recognised the distinction between
\(\overline{A_\mathrm{ZW}}\) and \(A_\mathrm{QS}\) and, deduced that
``the daily phases of waves 2 and 4-6 at 50\degree S must therefore be
random since the waves almost cancel themselves when added, whereas 1
and 3 must recur consistently in certain longitudes since they are
significantly large in the climatological mean''. This observation
motivates that stationary conditions in the circulation of the SH could
be measured using the quotient between the two quantities. As an analogy
with the constancy of the wind \citep{Singer1967}, the stationarity of
the QS can be estimated as

\begin{linenomath*}
\begin{equation}\label{eq:S}
\hat{S} = \frac{A_\mathrm{QS}}{\overline{A_\mathrm{ZW}}}
\end{equation}
\end{linenomath*}

It can be shown that \(\hat{S} = 1\) for completely stationary waves. On
the other hand, it can be demonstrated that the expected amplitude of
the sum of \(n\) waves with random phases and mean amplitude \(A\) is
\(An^{-1/2}\) \citep{Pain2005}. Thus, for completely non stationary
waves, the expected value of \(\hat{S}\) is \(n^{-1/2}\).

\begin{figure}[h]

{\centering \includegraphics{fig/QS-ZW/stationarity-1} 

}

\caption{Seasonal cycle of stationarity of the 300hPa geopotential QS2 computed using Equation \ref{eq:S} (shaded) and $\overline{A_\mathrm{ZW2}}$ (contours). From monthly NCEP/NCAR Reanalysis, 1958 to 2017.}\label{fig:stationarity}
\end{figure}

As an example, Figure \ref{fig:stationarity} shows \(\hat{S}\) for QS2
computed using Equation \ref{eq:S}. At the northern mid latitudes the
seasonal cycle of stationarity is similar to that described by
\(\overline{A_\mathrm{ZW}}\) (Figure \ref{fig:hurrell}) with maximum
values in boreal summer and minimum in the boreal winter. On the other
hand, the SH circulation shows a lower degree of QS2 stationarity than
that of the northern hemisphere or the tropics. At the SH is no clear
annual cycle and, even more, at 60\degree S, stationarity and
\(\overline{A_\mathrm{ZW}}\) appear to be anticorrelated.

\(\hat{S}\) can equivalently be mathematically defined as

\begin{linenomath*}
\begin{equation}\label{eq:S2}
\hat{S} =   \frac{\sum_t A_\mathrm{ZW}(t) \cos  \left [\alpha_\mathrm{zw}(t) - \alpha_{qs} \right ]}{\sum_t A_\mathrm{ZW}(t)}
\end{equation}
\end{linenomath*}

The numerator is the sum of the projections of each \(\mathrm{ZW}\) onto
the direction of the \(\mathrm{QS}\). Equation \ref{eq:S2} has some
advantages over Equation \ref{eq:S}. First, it makes is clear that
stationarity is a mixture of a phase effect and an amplitude effect.
Secondly, one can, in principle, replace \(\alpha_{qs}\) with any
direction of interest, allowing to evaluate \(\hat{S}(\alpha)\). This
can also be useful for removing variability due to the seasonal cycle.
The position of the monthly QS3 has a shift of about 15\degree between
January and July \citep{Loon1972}, so by replacing \(\alpha_{qs}\) with
\(\alpha_{qs}(month)\) (one for each month) one can evaluate
stationarity with respect to the seasonal changing position of the mean
wave. Finally, it is possible to transform the sums into running sums
with window \(w\) and obtain \(\hat{S}(w, t)\) and analyse variations of
stationarity with time.

While \(\hat{S}\) is used --sometimes as
\(2/\pi\arcsin \left (\hat{S} \right )\) \citep{Singer1967}-- in the
meteorological literature in the context of wind steadiness, to our
knowledge this is the first time it has been applied to the study of
atmospheric waves. However, its statistical properties are not well
studied. One problem with \(\hat{S}\) is that, as seen above, its
estimation from a finite sample has a positive bias, but its convergence
properties are not explored.

\subsection{Considerations about phase}

\label{sec:phase}

For defining local impacts, the phase of planetary waves is as important
as their amplitude if not more. One way of dealing with the phase of ZW
is to fix it. \citet{Yuan2008} use Principal Component Analysis on the
meridional wind field to obtain a spatial pattern of the leading mode
that is very similar to the QS3. The timeseries associated to this mode
is, then, an indication of the intensity of the ZW3 that is similar to
the QS3. A more direct approach is the index created by
\citet{Raphael2004}. Since it is based on the geopotential height
anomalies at the maximums of the QS3, it is sensitive to ZW3 patterns
with phase close to the stationary phase. An almost mathematically
equivalent approach (with correlation = 0.98) is to compute the
projection of each \(\mathrm{ZW}\) onto the direction of the
\(\mathrm{QS}\) (i.e.~the expression inside the sum of the numerator in
Equation \ref{eq:S2}). This methodology has fewer constrains in that the
phase of interest can be changed depending on the application.

\section{Conclusions}

The fact that zonal waves (ZW) and quasi-stationary waves (QS) are two
distinct but related phenomena has both practical and theoretical
implications.

First, researchers should be aware of which phenomena they want to study
and use the appropriate methods. The mean amplitude of the ZW could be
appropriate to study the vertical propagation of Rossby waves, for
example. But ZW amplitude could lead to misleading results if used as
the basis of local impacts studies because they are probably more
influenced by phase effects.

Secondly, comparison between results should also be made having this
issues in mind. For instance, \citet{Irving2015} compare their planetary
wave activity index with \citet{Raphael2004}'s wave 3 index and conclude
that the later cannot account for events with waves far removed from
their climatological position. However, in light of the discussion in
Section \ref{sec:phase}, this limitation becomes a feature, not a bug.

Although having a consistent nomenclature across papers is desirable, we
believe that this problems can be ameliorated by researchers detailing
their definitions and methodology. This is also good for clarity and
reproducibility. Since planetary waves are generally more stationary in
the northern hemisphere, these issues are more critical for studies of
the southern hemisphere.

Thirdly, the explorations of both ZW and QS can lead to novel levels of
analysis. Here, we showed it can be used to define a metric of
stationarity of quasi-stationary waves, but other applications are also
possible. \citet{Smith2012} used the phase relationship between ZW1 and
QS1 to show that linear interference between the QS1 and ZW1 was related
to vertical wave activity transport at the tropopause.

\emph{xx me falta un final acá xx}

\bibliography{qszw.bib}


\end{document}
